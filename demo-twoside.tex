%\documentclass[oneside,blindreview]{template/hdu_master_thesis} % 单面+盲审
\documentclass[twoside]{template/hdu_master_thesis} % 双面
\usepackage{biblatex}


\blindreviewhead{全日制专业型研究生} %全日制专业型研究生,全日制学术型研究生,...

\title{基于\LaTeX 的杭州电子科技大学研究生硕士论文模版}
\englishtitle{Master thesis template of Postgraduates of Hangzhou University of Electronic Science and technology based on latex}
\author{作者是我}
\englishauthor{I'm Author}

\supervisor{导师是我}
\englishsupervisor{I’m Supervisor}
\spvtitle{教授}
\englishspvtitle{Prof.}

\research{深度学习}
\major{计算机科技与计算}

\submitdate{2022年05月}
\defencedate{2022年03月}
\englishdefencedate{May 2022}

\graphicspath{{./figures/}}
\addbibresource{ref.bib}


\begin{document}
% 封面、中文题名页、英文题名页、独创声明和版权使用书 无页码
%\usetag{blindreview}% 要盲审风格启动这个,记得 注释“致谢.tex”和一些特殊信息(人名,学校,敏感信息)
\maketitle

\frontmatter
% 摘要部分
\include{contents/abstract_chinese.tex}
\include{contents/abstract_english.tex}

% 目录和术语表
\tableofcontents % 正文目录
\listoffigures   % 图目录,格式不正确,需要调整
\listoftables    % 表目录,格式不正确,需要调整
%\include{contents/denotation}% 术语及缩略词表(需要则开)

% 正文
\mainmatter
\include{contents/intro.tex} % 绪论
\include{contents/whyla.tex}
\include{contents/elem.tex}
\include{contents/sum.tex}

\include{contents/thanks.tex} % 致谢有人放到正文最后,有人放到附录
\backmatter

% 引用参考文献数据库
\printbibliography[heading=bibintoc]

% 附录部分
\appendix
% !TEX root = ../main.tex
\chapter{作者简历}

\noindent{一、教育经历:}

\begin{tabular}{llll}
    1999年11月至1999年12月: &  致远星高达学院  & 高达驾驶学徒  &  学生
\end{tabular}

\noindent{二、工作经历:}

\begin{tabular}{llll}
    1999年12月至2000年01月: &  解放三体远征军  & 高达高级驾驶远  &  士兵
\end{tabular}

\noindent{三、研究生期间专利成果:}

    \begin{enumerate}[{[1]}]
        \item{一种使用僵尸的永动机技术, 申请公布号:CN00000001A, 申请公布日:1111-11-11. (导师一作,本人二作)}
        \item{一种利用泥头车穿越异世界的技术, 申请公布号:CN00000002A, 申请公布日:2222-22-22.(导师二作,本人一作)}
    \end{enumerate}

\noindent{四、研究生期间论文成果:}

    \begin{enumerate}[{[1]}]
        \item{一种使用僵尸的永动机技术.(XXXX已投稿,导师通信,本人一作)}
        \item{一种利用泥头车穿越异世界的技术[J]. XXXX(会议/期刊), 2222, 1-1024.(顶会,已收录,本人一作)}
    \end{enumerate}
\end{document}