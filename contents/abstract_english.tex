% !TEX root = ../main.tex

% 定义英文摘要和关键字

\begin{eabstract}
The purpose of this paper is to differentiate between natural images (NI) acquired by digital cameras and computer- generated graphics (CG) created by computer graphics rendering software. The main contributions of this paper are threefold. First, we propose to utilize two different denoising filters for acquiring the first-order and second-order noise of the inspected image, and analyze its characteristics with assuming that residual noise follows the proposed statistical model. Second, under the framework of the hypothesis testing theory, the problem of identifying between NI and CG is smoothly transferred to the design of the likelihood ratio test (LRT) with knowing all the nuisance parameters, and meanwhile the performance of the LRT is theoretically investigated. Third, in the practical classification, using the estimated model parameters, we propose to establish a generalized likelihood ratio test (GLRT). A large scale of experimental results on simulated and real data directly verify that our proposed test has the ability of identifying CG from NI with high detection performance, and show the comparable effectiveness with some prior arts. Besides, the robustness of the proposed classifier is verified with considering the attacks generated by some post-processing techniques.
\end{eabstract}

\ekeywords{Natural image, computer-generated graphic, digital image forensics, statistical noise model, hypothesis testing}
